%% Generated by Sphinx.
\def\sphinxdocclass{report}
\documentclass[letterpaper,10pt,english]{sphinxmanual}
\ifdefined\pdfpxdimen
   \let\sphinxpxdimen\pdfpxdimen\else\newdimen\sphinxpxdimen
\fi \sphinxpxdimen=.75bp\relax
\ifdefined\pdfimageresolution
    \pdfimageresolution= \numexpr \dimexpr1in\relax/\sphinxpxdimen\relax
\fi
%% let collapsible pdf bookmarks panel have high depth per default
\PassOptionsToPackage{bookmarksdepth=5}{hyperref}

\PassOptionsToPackage{booktabs}{sphinx}
\PassOptionsToPackage{colorrows}{sphinx}

\PassOptionsToPackage{warn}{textcomp}
\usepackage[utf8]{inputenc}
\ifdefined\DeclareUnicodeCharacter
% support both utf8 and utf8x syntaxes
  \ifdefined\DeclareUnicodeCharacterAsOptional
    \def\sphinxDUC#1{\DeclareUnicodeCharacter{"#1}}
  \else
    \let\sphinxDUC\DeclareUnicodeCharacter
  \fi
  \sphinxDUC{00A0}{\nobreakspace}
  \sphinxDUC{2500}{\sphinxunichar{2500}}
  \sphinxDUC{2502}{\sphinxunichar{2502}}
  \sphinxDUC{2514}{\sphinxunichar{2514}}
  \sphinxDUC{251C}{\sphinxunichar{251C}}
  \sphinxDUC{2572}{\textbackslash}
\fi
\usepackage{cmap}
\usepackage[T1]{fontenc}
\usepackage{amsmath,amssymb,amstext}
\usepackage{babel}



\usepackage{tgtermes}
\usepackage{tgheros}
\renewcommand{\ttdefault}{txtt}



\usepackage[Bjarne]{fncychap}
\usepackage{sphinx}

\fvset{fontsize=auto}
\usepackage{geometry}


% Include hyperref last.
\usepackage{hyperref}
% Fix anchor placement for figures with captions.
\usepackage{hypcap}% it must be loaded after hyperref.
% Set up styles of URL: it should be placed after hyperref.
\urlstyle{same}

\addto\captionsenglish{\renewcommand{\contentsname}{Contents}}

\usepackage{sphinxmessages}
\setcounter{tocdepth}{1}



\title{pythonatomistics}
\date{Jul 29, 2023}
\release{0.1.0}
\author{Yavar}
\newcommand{\sphinxlogo}{\vbox{}}
\renewcommand{\releasename}{Release}
\makeindex
\begin{document}

\ifdefined\shorthandoff
  \ifnum\catcode`\=\string=\active\shorthandoff{=}\fi
  \ifnum\catcode`\"=\active\shorthandoff{"}\fi
\fi

\pagestyle{empty}
\sphinxmaketitle
\pagestyle{plain}
\sphinxtableofcontents
\pagestyle{normal}
\phantomsection\label{\detokenize{index::doc}}



\chapter{Introduction}
\label{\detokenize{index:introduction}}
\sphinxAtStartPar
Welcome to the “Python for Atomistic Simulation: Bridging Solid State and Quantum Chemistry” course!


\section{Course Overview}
\label{\detokenize{index:course-overview}}
\sphinxAtStartPar
This course is designed to provide learners with a comprehensive understanding of performing atomistic simulations using Python. The course covers two essential domains: solid\sphinxhyphen{}state materials using Quantum Espresso and quantum chemistry calculations using NWChem. By the end of this course, you will gain hands\sphinxhyphen{}on experience in using the Atomic Simulation Environment (ASE) library to set up and analyze DFT calculations for both solid\sphinxhyphen{}state and quantum chemistry applications.


\section{Course Objectives}
\label{\detokenize{index:course-objectives}}\begin{itemize}
\item {} 
\sphinxAtStartPar
Introduce the fundamental concepts of Density Functional Theory (DFT) and its relevance in atomistic simulations.

\item {} 
\sphinxAtStartPar
Familiarize learners with the Atomic Simulation Environment (ASE) and its Python API for DFT calculations.

\item {} 
\sphinxAtStartPar
Provide practical examples using Quantum Espresso for solid\sphinxhyphen{}state materials simulations.

\item {} 
\sphinxAtStartPar
Explore quantum chemistry calculations with NWChem for molecular properties and reactions.

\item {} 
\sphinxAtStartPar
Introduce some advanced tools to generate descriptors for machine learning based simulations.

\end{itemize}


\section{Who Should Take This Course?}
\label{\detokenize{index:who-should-take-this-course}}
\sphinxAtStartPar
This course is suitable for individuals interested in computational chemistry, materials science, and anyone looking to expand their knowledge of DFT simulations using Python. Basic knowledge of Python programming is beneficial, but not mandatory, as we will cover the necessary Python concepts throughout the course.


\section{Prerequisites}
\label{\detokenize{index:prerequisites}}\begin{itemize}
\item {} 
\sphinxAtStartPar
Basic understanding of Python programming (recommended, but not required).

\item {} 
\sphinxAtStartPar
Familiarity with fundamental chemistry and physics concepts.

\end{itemize}

\sphinxAtStartPar
Let’s get started on this exciting journey into the world of atomistic simulations with Python!

\begin{sphinxadmonition}{note}{Note:}
\sphinxAtStartPar
The examples in this course are provided with code snippets and step\sphinxhyphen{}by\sphinxhyphen{}step instructions. You can find the complete code and materials on GitHub repository for this course \sphinxurl{https://yavar-azar.github.io/pythonatomistics} .
\end{sphinxadmonition}

\sphinxstepscope


\subsection{Installing Ubuntu on VirtualBox}
\label{\detokenize{vbox/vbox:installing-ubuntu-on-virtualbox}}\label{\detokenize{vbox/vbox::doc}}

\subsubsection{Introduction}
\label{\detokenize{vbox/vbox:introduction}}
\sphinxAtStartPar
In this section, we will guide you through the process of installing Ubuntu on VirtualBox. VirtualBox is a powerful virtualization software that allows you to create and run virtual machines on your host operating system. Installing Ubuntu in a virtual machine enables you to practice the course material without affecting your main system.


\subsubsection{Step 1: Downloading Ubuntu ISO Image}
\label{\detokenize{vbox/vbox:step-1-downloading-ubuntu-iso-image}}
\sphinxAtStartPar
Before you begin, download the latest Ubuntu Desktop ISO image from the official website. Make sure to choose the appropriate version for your system, such as a 64\sphinxhyphen{}bit or 32\sphinxhyphen{}bit image.

\sphinxAtStartPar
\sphinxcode{\sphinxupquote{Ubuntu\_22.04.2}}


\subsubsection{Step 2: Creating a New Virtual Machine}
\label{\detokenize{vbox/vbox:step-2-creating-a-new-virtual-machine}}\begin{enumerate}
\sphinxsetlistlabels{\arabic}{enumi}{enumii}{}{.}%
\item {} 
\sphinxAtStartPar
Open VirtualBox and click on the “New” button to create a new virtual machine.

\item {} 
\sphinxAtStartPar
Give your virtual machine a name (e.g., “Ubuntu\_ASE”) and select “Linux” as the Type, and “Ubuntu (64\sphinxhyphen{}bit)” as the Version (or choose the appropriate version based on your ISO image).

\item {} 
\sphinxAtStartPar
Allocate memory to the virtual machine. We recommend at least 4GB for smooth performance, but you can adjust this based on your system’s resources.

\item {} 
\sphinxAtStartPar
Choose “Create a virtual hard disk now” and click “Create.”

\end{enumerate}


\subsubsection{Step 3: Installing Ubuntu on the Virtual Machine}
\label{\detokenize{vbox/vbox:step-3-installing-ubuntu-on-the-virtual-machine}}\begin{enumerate}
\sphinxsetlistlabels{\arabic}{enumi}{enumii}{}{.}%
\item {} 
\sphinxAtStartPar
In the VirtualBox Manager, select the newly created virtual machine and click on the “Start” button.

\item {} 
\sphinxAtStartPar
When prompted, browse and select the Ubuntu ISO image you downloaded earlier.

\item {} 
\sphinxAtStartPar
Follow the on\sphinxhyphen{}screen instructions to install Ubuntu on the virtual machine. You can choose the default options or customize the installation based on your preferences.

\end{enumerate}


\subsubsection{Step 4: Essential Post\sphinxhyphen{}Installation Setup}
\label{\detokenize{vbox/vbox:step-4-essential-post-installation-setup}}
\sphinxAtStartPar
After Ubuntu is installed on the virtual machine, you may need to perform some post\sphinxhyphen{}installation setup:
\begin{enumerate}
\sphinxsetlistlabels{\arabic}{enumi}{enumii}{}{.}%
\item {} 
\sphinxAtStartPar
Update Ubuntu: Open a terminal and run the following commands to update the system:

\begin{sphinxVerbatim}[commandchars=\\\{\}]
sudo\PYG{+w}{ }apt\PYG{+w}{ }update
sudo\PYG{+w}{ }apt\PYG{+w}{ }upgrade
\end{sphinxVerbatim}

\item {} 
\sphinxAtStartPar
Install Guest Additions: In the VirtualBox menu, go to “Devices” \sphinxhyphen{}\textgreater{} “Insert Guest Additions CD Image.” Then, open a terminal and run:

\begin{sphinxVerbatim}[commandchars=\\\{\}]
sudo\PYG{+w}{ }apt\PYG{+w}{ }install\PYG{+w}{ }build\PYGZhy{}essential\PYG{+w}{ }dkms
sudo\PYG{+w}{ }mount\PYG{+w}{ }/dev/cdrom\PYG{+w}{ }/media/cdrom
\PYG{n+nb}{cd}\PYG{+w}{ }/media/cdrom
sudo\PYG{+w}{ }./autorun.sh
\end{sphinxVerbatim}

\end{enumerate}


\subsubsection{Step 5: Install gcc and gfortran libraries}
\label{\detokenize{vbox/vbox:step-5-install-gcc-and-gfortran-libraries}}
\begin{sphinxVerbatim}[commandchars=\\\{\}]
sudo\PYG{+w}{ }apt\PYG{+w}{ }install\PYG{+w}{ }gcc\PYG{+w}{ }gfortran
\end{sphinxVerbatim}


\subsubsection{Conclusion}
\label{\detokenize{vbox/vbox:conclusion}}
\sphinxAtStartPar
Congratulations! You have successfully installed Ubuntu on VirtualBox. Your virtual machine is now ready to be used for the course. You can now proceed with the rest of the course content and practice your atomistic simulations with ease.

\sphinxAtStartPar
Remember to save your progress and take additional snapshots as you progress through the course to have checkpoints to revert to if needed.

\sphinxAtStartPar
Happy learning and experimenting with Python for Atomistic Simulation!

\sphinxstepscope


\subsection{Python for Atomistic Simulation}
\label{\detokenize{basics/basics:python-for-atomistic-simulation}}\label{\detokenize{basics/basics::doc}}

\subsubsection{Setting up a Virtual Environment}
\label{\detokenize{basics/basics:setting-up-a-virtual-environment}}
\sphinxAtStartPar
First, let’s set up a virtual environment using \sphinxtitleref{virtualenv}. If you haven’t installed \sphinxtitleref{virtualenv} yet, you can do it by running the following command:

\begin{sphinxVerbatim}[commandchars=\\\{\}]
pip\PYG{+w}{ }install\PYG{+w}{ }virtualenv
\end{sphinxVerbatim}

\sphinxAtStartPar
Once \sphinxtitleref{virtualenv} is installed, let’s create the virtual environment named “envase”:

\begin{sphinxVerbatim}[commandchars=\\\{\}]
virtualenv\PYG{+w}{ }envase
\end{sphinxVerbatim}

\sphinxAtStartPar
Next, activate the virtual environment (on macOS/Linux):

\begin{sphinxVerbatim}[commandchars=\\\{\}]
\PYG{n+nb}{source}\PYG{+w}{ }envase/bin/activate
\end{sphinxVerbatim}


\subsubsection{Installing ASE}
\label{\detokenize{basics/basics:installing-ase}}
\sphinxAtStartPar
Now that we have the virtual environment set up, let’s proceed to install ASE (Atomic Simulation Environment). We’ll use \sphinxtitleref{pip} to install it within the virtual environment:

\begin{sphinxVerbatim}[commandchars=\\\{\}]
pip\PYG{+w}{ }install\PYG{+w}{ }ase
\end{sphinxVerbatim}

\sphinxAtStartPar
ASE is now successfully installed in your virtual environment and ready to use!


\subsubsection{Part 3: Python Basics Review}
\label{\detokenize{basics/basics:part-3-python-basics-review}}
\sphinxAtStartPar
Let’s start with a brief review of some Python basics. We’ll cover the introduction to Python, variables and data types, control flow, loops, functions, and an overview of NumPy and Matplotlib.


\subsubsection{Introduction to Python}
\label{\detokenize{basics/basics:introduction-to-python}}
\sphinxAtStartPar
Python is a versatile, high\sphinxhyphen{}level programming language that’s easy to learn and widely used in various fields, including scientific computing and data analysis.


\paragraph{Variables and Data Types}
\label{\detokenize{basics/basics:variables-and-data-types}}
\sphinxAtStartPar
In Python, you can declare variables and assign values to them. Python is dynamically typed, meaning you don’t need to specify the data type explicitly.

\sphinxAtStartPar
Python Code for Variables and Data Types:

\begin{sphinxVerbatim}[commandchars=\\\{\}]
\PYG{c+c1}{\PYGZsh{} Variables and Data Types}
\PYG{n}{name} \PYG{o}{=} \PYG{l+s+s2}{\PYGZdq{}}\PYG{l+s+s2}{John}\PYG{l+s+s2}{\PYGZdq{}}
\PYG{n}{age} \PYG{o}{=} \PYG{l+m+mi}{25}
\PYG{n}{height} \PYG{o}{=} \PYG{l+m+mf}{1.75}
\PYG{n}{is\PYGZus{}student} \PYG{o}{=} \PYG{k+kc}{True}
\end{sphinxVerbatim}


\paragraph{Control Flow \sphinxhyphen{} Conditional Statements and Loops}
\label{\detokenize{basics/basics:control-flow-conditional-statements-and-loops}}
\sphinxAtStartPar
Python provides various control flow constructs, such as if\sphinxhyphen{}else statements and loops (for and while), to control the program’s flow based on conditions.

\sphinxAtStartPar
Python Code for Control Flow:

\begin{sphinxVerbatim}[commandchars=\\\{\}]
\PYG{c+c1}{\PYGZsh{} Control Flow}
\PYG{k}{if} \PYG{n}{age} \PYG{o}{\PYGZlt{}} \PYG{l+m+mi}{18}\PYG{p}{:}
    \PYG{n+nb}{print}\PYG{p}{(}\PYG{l+s+s2}{\PYGZdq{}}\PYG{l+s+s2}{You are a minor.}\PYG{l+s+s2}{\PYGZdq{}}\PYG{p}{)}
\PYG{k}{elif} \PYG{n}{age} \PYG{o}{\PYGZgt{}}\PYG{o}{=} \PYG{l+m+mi}{18} \PYG{o+ow}{and} \PYG{n}{age} \PYG{o}{\PYGZlt{}} \PYG{l+m+mi}{60}\PYG{p}{:}
    \PYG{n+nb}{print}\PYG{p}{(}\PYG{l+s+s2}{\PYGZdq{}}\PYG{l+s+s2}{You are an adult.}\PYG{l+s+s2}{\PYGZdq{}}\PYG{p}{)}
\PYG{k}{else}\PYG{p}{:}
    \PYG{n+nb}{print}\PYG{p}{(}\PYG{l+s+s2}{\PYGZdq{}}\PYG{l+s+s2}{You are a senior citizen.}\PYG{l+s+s2}{\PYGZdq{}}\PYG{p}{)}

\PYG{c+c1}{\PYGZsh{} Loops}
\PYG{k}{for} \PYG{n}{i} \PYG{o+ow}{in} \PYG{n+nb}{range}\PYG{p}{(}\PYG{l+m+mi}{5}\PYG{p}{)}\PYG{p}{:}
    \PYG{n+nb}{print}\PYG{p}{(}\PYG{l+s+sa}{f}\PYG{l+s+s2}{\PYGZdq{}}\PYG{l+s+s2}{Loop iteration: }\PYG{l+s+si}{\PYGZob{}}\PYG{n}{i}\PYG{l+s+si}{\PYGZcb{}}\PYG{l+s+s2}{\PYGZdq{}}\PYG{p}{)}

\PYG{c+c1}{\PYGZsh{} While Loop}
\PYG{n}{counter} \PYG{o}{=} \PYG{l+m+mi}{0}
\PYG{k}{while} \PYG{n}{counter} \PYG{o}{\PYGZlt{}} \PYG{l+m+mi}{5}\PYG{p}{:}
    \PYG{n+nb}{print}\PYG{p}{(}\PYG{l+s+sa}{f}\PYG{l+s+s2}{\PYGZdq{}}\PYG{l+s+s2}{While loop iteration: }\PYG{l+s+si}{\PYGZob{}}\PYG{n}{counter}\PYG{l+s+si}{\PYGZcb{}}\PYG{l+s+s2}{\PYGZdq{}}\PYG{p}{)}
    \PYG{n}{counter} \PYG{o}{+}\PYG{o}{=} \PYG{l+m+mi}{1}
\end{sphinxVerbatim}


\subsubsection{Functions}
\label{\detokenize{basics/basics:functions}}
\sphinxAtStartPar
Functions allow us to group a block of code and execute it whenever needed. They promote code reusability and modularity.

\sphinxAtStartPar
Python Code for Functions:

\begin{sphinxVerbatim}[commandchars=\\\{\}]
\PYG{c+c1}{\PYGZsh{} Functions}
\PYG{k}{def} \PYG{n+nf}{greet\PYGZus{}user}\PYG{p}{(}\PYG{n}{username}\PYG{p}{)}\PYG{p}{:}
    \PYG{n+nb}{print}\PYG{p}{(}\PYG{l+s+sa}{f}\PYG{l+s+s2}{\PYGZdq{}}\PYG{l+s+s2}{Hello, }\PYG{l+s+si}{\PYGZob{}}\PYG{n}{username}\PYG{l+s+si}{\PYGZcb{}}\PYG{l+s+s2}{! Welcome to our course.}\PYG{l+s+s2}{\PYGZdq{}}\PYG{p}{)}

\PYG{n}{greet\PYGZus{}user}\PYG{p}{(}\PYG{l+s+s2}{\PYGZdq{}}\PYG{l+s+s2}{Alice}\PYG{l+s+s2}{\PYGZdq{}}\PYG{p}{)}
\end{sphinxVerbatim}


\subsubsection{NumPy Basics}
\label{\detokenize{basics/basics:numpy-basics}}
\sphinxAtStartPar
NumPy is a fundamental library for numerical computing in Python. It provides support for large, multi\sphinxhyphen{}dimensional arrays and matrices, along with an extensive collection of high\sphinxhyphen{}level mathematical functions to operate on these arrays.

\sphinxAtStartPar
Python Code for NumPy Basics:

\begin{sphinxVerbatim}[commandchars=\\\{\}]
\PYG{k+kn}{import} \PYG{n+nn}{numpy} \PYG{k}{as} \PYG{n+nn}{np}

\PYG{c+c1}{\PYGZsh{} Creating arrays}
\PYG{n}{arr1} \PYG{o}{=} \PYG{n}{np}\PYG{o}{.}\PYG{n}{array}\PYG{p}{(}\PYG{p}{[}\PYG{l+m+mi}{1}\PYG{p}{,} \PYG{l+m+mi}{2}\PYG{p}{,} \PYG{l+m+mi}{3}\PYG{p}{,} \PYG{l+m+mi}{4}\PYG{p}{,} \PYG{l+m+mi}{5}\PYG{p}{]}\PYG{p}{)}
\PYG{n}{arr2} \PYG{o}{=} \PYG{n}{np}\PYG{o}{.}\PYG{n}{arange}\PYG{p}{(}\PYG{l+m+mi}{10}\PYG{p}{,} \PYG{l+m+mi}{21}\PYG{p}{,} \PYG{l+m+mi}{2}\PYG{p}{)}
\PYG{n}{arr3} \PYG{o}{=} \PYG{n}{np}\PYG{o}{.}\PYG{n}{zeros}\PYG{p}{(}\PYG{p}{(}\PYG{l+m+mi}{2}\PYG{p}{,} \PYG{l+m+mi}{3}\PYG{p}{)}\PYG{p}{)}
\PYG{n}{arr4} \PYG{o}{=} \PYG{n}{np}\PYG{o}{.}\PYG{n}{ones}\PYG{p}{(}\PYG{p}{(}\PYG{l+m+mi}{3}\PYG{p}{,} \PYG{l+m+mi}{2}\PYG{p}{)}\PYG{p}{)}

\PYG{c+c1}{\PYGZsh{} Array operations}
\PYG{n}{sum\PYGZus{}array} \PYG{o}{=} \PYG{n}{arr1} \PYG{o}{+} \PYG{n}{arr2}
\PYG{n}{dot\PYGZus{}product} \PYG{o}{=} \PYG{n}{np}\PYG{o}{.}\PYG{n}{dot}\PYG{p}{(}\PYG{n}{arr3}\PYG{p}{,} \PYG{n}{arr4}\PYG{p}{)}
\end{sphinxVerbatim}


\subsubsection{Introduction to Matplotlib}
\label{\detokenize{basics/basics:introduction-to-matplotlib}}
\sphinxAtStartPar
Matplotlib is a widely\sphinxhyphen{}used library for creating static, interactive, and animated plots in Python. It enables data visualization with a wide range of customization options.

\sphinxAtStartPar
Python Code for Matplotlib:

\begin{sphinxVerbatim}[commandchars=\\\{\}]
\PYG{k+kn}{import} \PYG{n+nn}{matplotlib}\PYG{n+nn}{.}\PYG{n+nn}{pyplot} \PYG{k}{as} \PYG{n+nn}{plt}

\PYG{c+c1}{\PYGZsh{} Creating simple plots}
\PYG{n}{x} \PYG{o}{=} \PYG{n}{np}\PYG{o}{.}\PYG{n}{linspace}\PYG{p}{(}\PYG{l+m+mi}{0}\PYG{p}{,} \PYG{l+m+mi}{10}\PYG{p}{,} \PYG{l+m+mi}{100}\PYG{p}{)}
\PYG{n}{y} \PYG{o}{=} \PYG{n}{np}\PYG{o}{.}\PYG{n}{sin}\PYG{p}{(}\PYG{n}{x}\PYG{p}{)}
\PYG{n}{plt}\PYG{o}{.}\PYG{n}{plot}\PYG{p}{(}\PYG{n}{x}\PYG{p}{,} \PYG{n}{y}\PYG{p}{)}
\PYG{n}{plt}\PYG{o}{.}\PYG{n}{xlabel}\PYG{p}{(}\PYG{l+s+s2}{\PYGZdq{}}\PYG{l+s+s2}{x\PYGZhy{}axis}\PYG{l+s+s2}{\PYGZdq{}}\PYG{p}{)}
\PYG{n}{plt}\PYG{o}{.}\PYG{n}{ylabel}\PYG{p}{(}\PYG{l+s+s2}{\PYGZdq{}}\PYG{l+s+s2}{y\PYGZhy{}axis}\PYG{l+s+s2}{\PYGZdq{}}\PYG{p}{)}
\PYG{n}{plt}\PYG{o}{.}\PYG{n}{title}\PYG{p}{(}\PYG{l+s+s2}{\PYGZdq{}}\PYG{l+s+s2}{Sine Function}\PYG{l+s+s2}{\PYGZdq{}}\PYG{p}{)}
\PYG{n}{plt}\PYG{o}{.}\PYG{n}{grid}\PYG{p}{(}\PYG{k+kc}{True}\PYG{p}{)}
\PYG{n}{plt}\PYG{o}{.}\PYG{n}{show}\PYG{p}{(}\PYG{p}{)}
\end{sphinxVerbatim}


\subsubsection{Conclusion}
\label{\detokenize{basics/basics:conclusion}}
\sphinxAtStartPar
Congratulations! You’ve completed the Python basics review and set up the ASE environment within your virtual environment. In the next section, we’ll delve deeper into atomistic simulations with ASE and Python.

\sphinxAtStartPar
Remember to activate the virtual environment whenever you work on the course materials related to ASE to ensure a clean and isolated environment for your simulations.

\sphinxAtStartPar
Happy learning and happy experimenting with Python for Atomistic Simulation!


\chapter{Indices and tables}
\label{\detokenize{index:indices-and-tables}}\begin{itemize}
\item {} 
\sphinxAtStartPar
\DUrole{xref,std,std-ref}{genindex}

\item {} 
\sphinxAtStartPar
\DUrole{xref,std,std-ref}{modindex}

\item {} 
\sphinxAtStartPar
\DUrole{xref,std,std-ref}{search}

\end{itemize}



\renewcommand{\indexname}{Index}
\printindex
\end{document}